
%% bare_conf.tex
%% V1.4b
%% 2015/08/26
%% by Michael Shell
%% See:
%% http://www.michaelshell.org/
%% for current contact information.
%%
%% This is a skeleton file demonstrating the use of IEEEtran.cls
%% (requires IEEEtran.cls version 1.8b or later) with an IEEE
%% conference paper.
%%
%% Support sites:
%% http://www.michaelshell.org/tex/ieeetran/
%% http://www.ctan.org/pkg/ieeetran
%% and
%% http://www.ieee.org/

%%*************************************************************************
%% Legal Notice:
%% This code is offered as-is without any warranty either expressed or
%% implied; without even the implied warranty of MERCHANTABILITY or
%% FITNESS FOR A PARTICULAR PURPOSE! 
%% User assumes all risk.
%% In no event shall the IEEE or any contributor to this code be liable for
%% any damages or losses, including, but not limited to, incidental,
%% consequential, or any other damages, resulting from the use or misuse
%% of any information contained here.
%%
%% All comments are the opinions of their respective authors and are not
%% necessarily endorsed by the IEEE.
%%
%% This work is distributed under the LaTeX Project Public License (LPPL)
%% ( http://www.latex-project.org/ ) version 1.3, and may be freely used,
%% distributed and modified. A copy of the LPPL, version 1.3, is included
%% in the base LaTeX documentation of all distributions of LaTeX released
%% 2003/12/01 or later.
%% Retain all contribution notices and credits.
%% ** Modified files should be clearly indicated as such, including  **
%% ** renaming them and changing author support contact information. **
%%*************************************************************************


% *** Authors should verify (and, if needed, correct) their LaTeX system  ***
% *** with the testflow diagnostic prior to trusting their LaTeX platform ***
% *** with production work. The IEEE's font choices and paper sizes can   ***
% *** trigger bugs that do not appear when using other class files.       ***                          ***
% The testflow support page is at:
% http://www.michaelshell.org/tex/testflow/



\documentclass[conference]{IEEEtran}
% Some Computer Society conferences also require the compsoc mode option,
% but others use the standard conference format.
%
% If IEEEtran.cls has not been installed into the LaTeX system files,
% manually specify the path to it like:
% \documentclass[conference]{../sty/IEEEtran}





% Some very useful LaTeX packages include:
% (uncomment the ones you want to load)


% *** MISC UTILITY PACKAGES ***
%
%\usepackage{ifpdf}
% Heiko Oberdiek's ifpdf.sty is very useful if you need conditional
% compilation based on whether the output is pdf or dvi.
% usage:
% \ifpdf
%   % pdf code
% \else
%   % dvi code
% \fi
% The latest version of ifpdf.sty can be obtained from:
% http://www.ctan.org/pkg/ifpdf
% Also, note that IEEEtran.cls V1.7 and later provides a builtin
% \ifCLASSINFOpdf conditional that works the same way.
% When switching from latex to pdflatex and vice-versa, the compiler may
% have to be run twice to clear warning/error messages.






% *** CITATION PACKAGES ***
%
%\usepackage{cite}
% cite.sty was written by Donald Arseneau
% V1.6 and later of IEEEtran pre-defines the format of the cite.sty package
% \cite{} output to follow that of the IEEE. Loading the cite package will
% result in citation numbers being automatically sorted and properly
% "compressed/ranged". e.g., [1], [9], [2], [7], [5], [6] without using
% cite.sty will become [1], [2], [5]--[7], [9] using cite.sty. cite.sty's
% \cite will automatically add leading space, if needed. Use cite.sty's
% noadjust option (cite.sty V3.8 and later) if you want to turn this off
% such as if a citation ever needs to be enclosed in parenthesis.
% cite.sty is already installed on most LaTeX systems. Be sure and use
% version 5.0 (2009-03-20) and later if using hyperref.sty.
% The latest version can be obtained at:
% http://www.ctan.org/pkg/cite
% The documentation is contained in the cite.sty file itself.






% *** GRAPHICS RELATED PACKAGES ***
%
\ifCLASSINFOpdf
  % \usepackage[pdftex]{graphicx}
  % declare the path(s) where your graphic files are
  % \graphicspath{{../pdf/}{../jpeg/}}
  % and their extensions so you won't have to specify these with
  % every instance of \includegraphics
  % \DeclareGraphicsExtensions{.pdf,.jpeg,.png}
\else
  % or other class option (dvipsone, dvipdf, if not using dvips). graphicx
  % will default to the driver specified in the system graphics.cfg if no
  % driver is specified.
  % \usepackage[dvips]{graphicx}
  % declare the path(s) where your graphic files are
  % \graphicspath{{../eps/}}
  % and their extensions so you won't have to specify these with
  % every instance of \includegraphics
  % \DeclareGraphicsExtensions{.eps}
\fi
% graphicx was written by David Carlisle and Sebastian Rahtz. It is
% required if you want graphics, photos, etc. graphicx.sty is already
% installed on most LaTeX systems. The latest version and documentation
% can be obtained at: 
% http://www.ctan.org/pkg/graphicx
% Another good source of documentation is "Using Imported Graphics in
% LaTeX2e" by Keith Reckdahl which can be found at:
% http://www.ctan.org/pkg/epslatex
%
% latex, and pdflatex in dvi mode, support graphics in encapsulated
% postscript (.eps) format. pdflatex in pdf mode supports graphics
% in .pdf, .jpeg, .png and .mps (metapost) formats. Users should ensure
% that all non-photo figures use a vector format (.eps, .pdf, .mps) and
% not a bitmapped formats (.jpeg, .png). The IEEE frowns on bitmapped formats
% which can result in "jaggedy"/blurry rendering of lines and letters as
% well as large increases in file sizes.
%
% You can find documentation about the pdfTeX application at:
% http://www.tug.org/applications/pdftex





% *** MATH PACKAGES ***
%
%\usepackage{amsmath}
% A popular package from the American Mathematical Society that provides
% many useful and powerful commands for dealing with mathematics.
%
% Note that the amsmath package sets \interdisplaylinepenalty to 10000
% thus preventing page breaks from occurring within multiline equations. Use:
%\interdisplaylinepenalty=2500
% after loading amsmath to restore such page breaks as IEEEtran.cls normally
% does. amsmath.sty is already installed on most LaTeX systems. The latest
% version and documentation can be obtained at:
% http://www.ctan.org/pkg/amsmath





% *** SPECIALIZED LIST PACKAGES ***
%
%\usepackage{algorithmic}
% algorithmic.sty was written by Peter Williams and Rogerio Brito.
% This package provides an algorithmic environment fo describing algorithms.
% You can use the algorithmic environment in-text or within a figure
% environment to provide for a floating algorithm. Do NOT use the algorithm
% floating environment provided by algorithm.sty (by the same authors) or
% algorithm2e.sty (by Christophe Fiorio) as the IEEE does not use dedicated
% algorithm float types and packages that provide these will not provide
% correct IEEE style captions. The latest version and documentation of
% algorithmic.sty can be obtained at:
% http://www.ctan.org/pkg/algorithms
% Also of interest may be the (relatively newer and more customizable)
% algorithmicx.sty package by Szasz Janos:
% http://www.ctan.org/pkg/algorithmicx




% *** ALIGNMENT PACKAGES ***
%
%\usepackage{array}
% Frank Mittelbach's and David Carlisle's array.sty patches and improves
% the standard LaTeX2e array and tabular environments to provide better
% appearance and additional user controls. As the default LaTeX2e table
% generation code is lacking to the point of almost being broken with
% respect to the quality of the end results, all users are strongly
% advised to use an enhanced (at the very least that provided by array.sty)
% set of table tools. array.sty is already installed on most systems. The
% latest version and documentation can be obtained at:
% http://www.ctan.org/pkg/array


% IEEEtran contains the IEEEeqnarray family of commands that can be used to
% generate multiline equations as well as matrices, tables, etc., of high
% quality.




% *** SUBFIGURE PACKAGES ***
%\ifCLASSOPTIONcompsoc
%  \usepackage[caption=false,font=normalsize,labelfont=sf,textfont=sf]{subfig}
%\else
%  \usepackage[caption=false,font=footnotesize]{subfig}
%\fi
% subfig.sty, written by Steven Douglas Cochran, is the modern replacement
% for subfigure.sty, the latter of which is no longer maintained and is
% incompatible with some LaTeX packages including fixltx2e. However,
% subfig.sty requires and automatically loads Axel Sommerfeldt's caption.sty
% which will override IEEEtran.cls' handling of captions and this will result
% in non-IEEE style figure/table captions. To prevent this problem, be sure
% and invoke subfig.sty's "caption=false" package option (available since
% subfig.sty version 1.3, 2005/06/28) as this is will preserve IEEEtran.cls
% handling of captions.
% Note that the Computer Society format requires a larger sans serif font
% than the serif footnote size font used in traditional IEEE formatting
% and thus the need to invoke different subfig.sty package options depending
% on whether compsoc mode has been enabled.
%
% The latest version and documentation of subfig.sty can be obtained at:
% http://www.ctan.org/pkg/subfig




% *** FLOAT PACKAGES ***
%
%\usepackage{fixltx2e}
% fixltx2e, the successor to the earlier fix2col.sty, was written by
% Frank Mittelbach and David Carlisle. This package corrects a few problems
% in the LaTeX2e kernel, the most notable of which is that in current
% LaTeX2e releases, the ordering of single and double column floats is not
% guaranteed to be preserved. Thus, an unpatched LaTeX2e can allow a
% single column figure to be placed prior to an earlier double column
% figure.
% Be aware that LaTeX2e kernels dated 2015 and later have fixltx2e.sty's
% corrections already built into the system in which case a warning will
% be issued if an attempt is made to load fixltx2e.sty as it is no longer
% needed.
% The latest version and documentation can be found at:
% http://www.ctan.org/pkg/fixltx2e


%\usepackage{stfloats}
% stfloats.sty was written by Sigitas Tolusis. This package gives LaTeX2e
% the ability to do double column floats at the bottom of the page as well
% as the top. (e.g., "\begin{figure*}[!b]" is not normally possible in
% LaTeX2e). It also provides a command:
%\fnbelowfloat
% to enable the placement of footnotes below bottom floats (the standard
% LaTeX2e kernel puts them above bottom floats). This is an invasive package
% which rewrites many portions of the LaTeX2e float routines. It may not work
% with other packages that modify the LaTeX2e float routines. The latest
% version and documentation can be obtained at:
% http://www.ctan.org/pkg/stfloats
% Do not use the stfloats baselinefloat ability as the IEEE does not allow
% \baselineskip to stretch. Authors submitting work to the IEEE should note
% that the IEEE rarely uses double column equations and that authors should try
% to avoid such use. Do not be tempted to use the cuted.sty or midfloat.sty
% packages (also by Sigitas Tolusis) as the IEEE does not format its papers in
% such ways.
% Do not attempt to use stfloats with fixltx2e as they are incompatible.
% Instead, use Morten Hogholm'a dblfloatfix which combines the features
% of both fixltx2e and stfloats:
%
% \usepackage{dblfloatfix}
% The latest version can be found at:
% http://www.ctan.org/pkg/dblfloatfix




% *** PDF, URL AND HYPERLINK PACKAGES ***
%
%\usepackage{url}
% url.sty was written by Donald Arseneau. It provides better support for
% handling and breaking URLs. url.sty is already installed on most LaTeX
% systems. The latest version and documentation can be obtained at:
% http://www.ctan.org/pkg/url
% Basically, \url{my_url_here}.




% *** Do not adjust lengths that control margins, column widths, etc. ***
% *** Do not use packages that alter fonts (such as pslatex).         ***
% There should be no need to do such things with IEEEtran.cls V1.6 and later.
% (Unless specifically asked to do so by the journal or conference you plan
% to submit to, of course. )


% correct bad hyphenation here
\hyphenation{op-tical net-works semi-conduc-tor}


\begin{document}
%
% paper title
% Titles are generally capitalized except for words such as a, an, and, as,
% at, but, by, for, in, nor, of, on, or, the, to and up, which are usually
% not capitalized unless they are the first or last word of the title.
% Linebreaks \\ can be used within to get better formatting as desired.
% Do not put math or special symbols in the title.
\title{Marco Polo:\\ Paired Robotic Hunting}


% author names and affiliations
% use a multiple column layout for up to three different
% affiliations
\author{\IEEEauthorblockN{Anupriya Agarwal}
\IEEEauthorblockA{School of Computer Science\\
The University of Alabama\\
Tuscaloosa, Alabama 35404}
\and
\IEEEauthorblockN{Christian Brewton}
\IEEEauthorblockA{School of Computer Science\\
The University of Alabama\\
Tuscaloosa, Alabama 35404}
\and
\IEEEauthorblockN{Christopher Popovich}
\IEEEauthorblockA{School of Computer Science\\
The University of Alabama\\
Tuscaloosa, Alabama 35404}
\and
\IEEEauthorblockN{William Hampton}
\IEEEauthorblockA{School of Computer Science\\
The University of Alabama\\
Tuscaloosa, Alabama 35404}
}

% make the title area
\maketitle

% As a general rule, do not put math, special symbols or citations
% in the abstract
\begin{abstract}
We implement an algorithm for two robots to cooperatively hunt and capture a moving target.
The hunters are given their distance to the target at regular intervals.
They use the data to identify the target's location and path, then approach from both sides.
We compare the alogorithm to a naive approach, where the hunters move directly towards the target's last known location.
\end{abstract}

% no keywords

\section{Introduction}
Our project is a variation on Marco Polo, a game in which one player attempts to find another using auditory cues.
We are interested in a variation of this game with two AI-controlled hunters and a human-controlled target.
We will be comparing a simple cooperative algorithm for the hunters to a naive, direct approach.

The two hunters and their prey will be set in a rectangular arena with obstacles scattered around.
Periodically, the hunters will be told the distance between them and their prey.
The hunters can locate the prey by triangulation, but will be required to path around obstacles and trap the prey.
Capturing will be done by reaching the target within a small distance, to avoid collisions.

Algorithms for intercepting a moving target are of interest in many different areas.
They range from security response and tracking wildlife to soccer and tracking inanimate objects.
Our project would be most useful as a starting point for a problem involving a small number of mobile intercepting robots.
Research is available for interception of a single target using a mobile robot, but there is significantly less data on intercepting with multiple robots.
Additionally, many papers assume that the target is tracked using visual sensors, which gives more information to the hunters than our problem does.

\section{Related Work}

\subsection{Single Robot Tracking}
The majority of algorithms for intercepting a moving target assume a single hunter robot.
This approach can be seen in ``Vision-based interception'' \cite{freda}.
This paper proposes the problem of tracking a ball using visual sensors on a mobile robot.
A large part of the focus is on keeping the ball within the camera's view for easier tracking.
The algorithm is very successful in tracking and intercepting the ball, but the assumptions made in the paper are significantly different from ours \cite{freda}.
An important difference is the assumption of visual contact, which gives more information to the hunter and requires more data processing.
The introduction of obstacles and a target that moves less linearly would likely cause the algorithm to fail.

\subsection{Multiple Robot Exploration}
An article ``Multi-Robot Collaboration for Robust Exploration'' perhaps provides a more useful algorithm for our problem \cite{rekleitis}.
It proposes a two-robot localization and mapping system, which would prove useful for dealing with obstacles not on the map.
Although there is no target for the robots, the cooperative abilities of the algorithm are interesting.
The robots attempt to explore the most area while still remaining in range of each other \cite{rekleitis}.
The general approach may be appropriate for capturing a target, but the specific algorithm will not work.
However, if we required our hunters to map the area themselves, the algorithm would find use as an initial phase.

\section{Approach}

\subsection{Methods}
Although other approaches have explored topics closely related to our problem, they are not quite the same.
We intend to implement a simple and low cost algorithm that still predicts and pursues the target intelligently.
The hunters will not have visual contact with the target, and will need to navigate around multiple obstacles to find the target.
The most important difference is the addition of simple cooperation between the two hunter robots.
Although the problem was inspired by a simple game, it could still yield interesting results for this particular niche.
We intend to make use of the ``move base'' package in ROS to allow for a relatively simple algorithmic approach.

\subsection{Locating the Prey}
The hunters will be given the distance from each of them to the prey.
By using triangulation, the position of the prey can be narrowed down to two possibilities.
Then, by checking the most recent measurements, the hunters can decide which position the prey is in.
We expect the position to be located quickly and with very few measurements.
If the location is not found quickly enough in our testing, we may have each hunter move towards a different possible location.
This would find the correct location by noting which robot is moving closer to the target.
The problem can be made a bit more complex by adding noise.

\subsection{Trapping the Prey}
We will have two algorithms: one naive, and one complex.
The naive algorithm will have both hunters pathing directly towards the last known position of the prey.
The hunters will only cooperate in that they will avoid bumping into each other.
We expect this to be inefficient, and on larger maps the hunters could fail to capture the target in a reasonable amount of time.
Aside from locating the prey, we will only need ensure the robots can sense each other to avoid collisions.

The worldy algorithm will have the two hunters attempt to surround the prey and approach from opposite sides.
Once the location of the target has been found, further measurements easily give a straight line prediction of the target's velocity.
Our algorithm will predict the target's future location, and have one hunter move ahead of that position.
The other hunter will be directed to move behind the target.
The robots will move progressively closer, eventually reaching capturing distance to the prey.
We expect this algorithm to significantly improve on the naive algorithm.

\subsection{Additional Considerations}
For an unbiased comparison between the two algorithms, we will have the human move in a repeated pattern for both trials.
However, an additional challenge we are interested in is how well an intelligent human can avoid the hunters.
We would like to attempt to adjust the algorithm for this purpose, but needed modifications are not likely to become apparent until we implement the algorithm.
The adjustments we can make in the time available to us are limited to minor modifications of the ``surround and close in'' algorithm.
It may, for example, be useful for the robot in front of the target to remain still while the other hunter closes in.
If there is an object between the foremost hunter and the target, remaining still momentarily would force the human to choose which direction to go first.
Another possibility is to have the hunters avoid moving too far from each other.
This could make the algorithm perform worse, but it would make it difficult for the target to pass between the robots.
It requires modifying the algorithm in a nontrivial way, but it might also cause behavior as if a net were stretched between the hunters.
We will most likely not implement this unless our algorithm is too simplistic for the final project.

% references section
\bibliographystyle{IEEEtran}
\bibliography{IEEEabrv,final}

% that's all folks
\end{document}


